% !TEX root = ../main.tex
%
\pdfbookmark[0]{Abstract}{Abstract}
\chapter*{Abstract}
\label{sec:abstract}
\vspace*{-10mm}

In this thesis the results of the search of the \dst dibaryon at the CERN Large Hadron Collider
(LHC) with the ALICE experiment, are presented.

The \dst dibaryon is an exotic state predicted by Dyson and Xuong in 1964, studying the SU(6)
flavour symmetry. In the 70s and 80s many theoretical studies shown that this exotic state 
should be strongly bounded and experimentally observable.
In the same years different experiments searched the \dst dibaryon \ -- and many other dibaryons
-- \ however, no experiments provided indisputable evidence of its existence.
In 2011 the WASA-at-COSY collaboration reported the observation of a narrow resonance,
compatible with the \dst predicted properties, studying the double-pionic fusion reaction 
$pn \rightarrow d+\pi^{0}+\pi^{0}$ cross section.
Further studies conducted by the WASA experiment observed the same resonance in other channel,
as an example $pn \rightarrow d+\pip+\pim$.

The WASA observation requires an independent experimental confirmation. The aim of this thesis,
therefore, is to study the possibility to observe the \dst dibaryon with the ALICE experiment.
The eventual observation of this state would be the first confirmation of the existence of a 
non-trivial dibaryon.
The research was conducted analysing data related to \pPb collisions at \sctev collected by the ALICE
experiment in 2016.

La ricerca è stata condotta sui dati relativi alle collisioni p-Pb a \sqrt{s_{NN}} = 5.02 TeV raccolti dall’esperimento ALICE nel 2016.
La strategia adottata per la ricerca di d*(2380) è stata quella di un analisi cieca della distribuzione di massa invariante dei prodotti del decadimento di d*(2380) nel canale d*(2380) \rightarrow d+\pi^{+}+\pi^{-}. 
Come prima cosa è stata studiata l’efficienza di ricostruzione del decadimento con i detector di ALICE utilizzando delle simulazioni Monte Carlo. Per far questo diverse configurazioni dei detector usati per l’identificazione delle particelle sono stati presi in considerazione in modo da trovare la configurazione che garantisce le migliori performance di ricostruzione.
Successivamente è stato studiato il fondo delle misure. Per far questo sono stati sviluppati due modelli indipendenti, entrambi basati sulla tecnica dell’event mixing, in grado di riprodurre il fondo della misura con un incertezza relativa inferiore al 5 %. 
Basandosi sul modello di adronizzazione statistica è stata stimata la produzione di d*(2380) attesa nel campione di dati considerato, trovando una produzione attesa di 10^{-5} d*(2380) per evento.
La stima della produzione di d*(2380) attesa è stata utilizzata per stimare la significatività statistica della misura.