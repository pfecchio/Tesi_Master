% !TEX root = ../main.tex
%
\pdfbookmark[0]{Abstract}{Abstract}
\chapter*{Abstract}
\label{sec:abstract}
\vspace*{-10mm}

In this thesis the results of the search of the \dst dibaryon at the CERN Large Hadron Collider
(LHC) with the ALICE experiment, are presented.

The \dst dibaryon is an exotic state predicted by Dyson and Xuong in 1964, studying the SU(6)
flavour symmetry. In the 70s and 80s many theoretical studies shown that this exotic state 
should be strongly bounded and experimentally observable.
In the same years different experiments searched the \dst dibaryon \ -- and many other dibaryons
-- \ however, no experiments provided indisputable evidence of its existence.
In 2011 the WASA-at-COSY collaboration reported the observation of a narrow resonance,
compatible with the \dst predicted properties, studying the double-pionic fusion reaction 
$pn \rightarrow d+\pi^{0}+\pi^{0}$ cross section.
Further studies conducted by the WASA experiment observed the same resonance in other channel,
as an example $pn \rightarrow d+\pip+\pim$.

The WASA observation requires an independent experimental confirmation. The aim of this thesis,
therefore, is to study the possibility to observe the \dst dibaryon with the ALICE experiment.
The eventual observation of this state would be the first confirmation of the existence of a 
non-trivial dibaryon.

The research was conducted analysing data related to \pPb collisions at \sctev collected by the ALICE
experiment in 2016.
The analysis strategy adopted for the search of the \dst was a blind analysis of the invariant mass
distribution of the \dst decay product in the \dstdecay decay channel.

Firstly, the reconstruction efficiency of the considered decay for the ALICE apparatus was 
estimated in dedicated Monte Carlo productions. The efficiency was studied in three different
configurations of the detectors used for the particle identifications, in order to determine
which configuration ensure highest reconstruction efficiency.
Then, the background of the measurement was studied. Two different models, both based on the
event mixing technique, was developed in order to describe the background. Both models are able to
reproduce the background of the measurement with relative uncertainties < 5\%.
The expected \dst production yield in \pPb collisions at \sctev, was estimated according to the 
statistical hadronization model, finding an expected yield of $10^{-5}$.
The production yield estimation, together with the background models was used to estimate the
expected statistical significance of the measurement, which results to be around 0.25 $\sigma$.
Finally, the measured invariant mass distribution of the \dst decay product was compared with 
one of the background models in order to verify if the \dst peak is visible in the data.
Nevertheless, the measured invariant mass distribution is statistically consistent with the
background model, thus no \dst signal was observed.

Since it was not possible to observe the \dst dibaryon in the studied data sample, an upper limit
on the \dst production in \pPb collisions at \sctev was set. Because of the low statistical 
significance of the measurement, the limit does not exclude that the \dst could be produced 
according to the statistical hadronization model.
Lastly, the number of \pPb collisions needed to achieve 5$\sigma$ significance of the measurement 
was estimated providing a prediction for future analysis in the High Luminosity LHC era.
