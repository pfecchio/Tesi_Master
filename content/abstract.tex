% !TEX root = ../main.tex
%
\pdfbookmark[0]{Abstract}{Abstract}
\chapter*{Abstract}
\label{sec:abstract}
\vspace*{-10mm}

In the ultra-relativistic heavy ion collisions at the CERN Large Hadron Collider (LHC), a state of
matter, called Quark Gluon Plasma (QGP), is created.
A typical signature of a heavy ion collision related to the production of the QGP is the large
number of particles produced ($dN_{ch}/d\eta$ up to 2000 in \PbPb collisions at \sctev).
This high multiplicity environment is a big experimental challenge, since the experiments have to 
cope with a high density of signals in their sensitive volume.
A Large Ion Collider Experiment (ALICE) has been designed to deal with such environment in order to
study in details the characteristics of the QGP.
% Among the several particles produced in a heavy ion collision, light (anti- )hypernuclei are of special interest since the production mechanism of such loosely bound states is not clear at present in high energy collisions. In particular, this thesis is focused on the study of the production and the measurement of the lifetime of the lightest known Λ-hypernucleus, the hypertriton (3ΛH) which is a bound state of a proton, a neutron and a Λ. The production rate at the LHC for this light hypernucleus is of the or- der of one every ten thousands Pb–Pb collisions with the highest charged particle density.

