% !TEX root = ../main.tex
%
\chapter{Dibaryons} \label{sec:2}


The term dibaryon denotes any object with baryon number B = 2. In this sense the first known dibaryon
has been the deuteron discovered in 1932 ~\cite{deu_disc}.
In terms of quarks a dibaryon is composed state of six valence quarks.
It may be of molecular type, i.e. spatially extended with two well separated interacting quark 
bags as is the case of the deuteron. Or \ -- more exotic and hence more interesting -- \ a dibaryon
could be a spatially compact hexaquark object, where all six quarks are confinde in a single quark bag.

The dibaryon searches have started in the fifties and have been a very changeful one with many 
successes and failures. 
Early predictions of a vast number of dibaryon states initiated endless experimental claims,
but finally none survived careful experimental investigations. 

Despite their long history dibaryon searches have recently received renewed interest,
in particular by the recognition that there are more complex quark configurations than just the 
well known $q\bar{q}$ and $qqq$ systems.
The WASA-at-COSY collaboration has found that the double-pionic fusion reaction $pn \rightarrow 
d \pi^{0} \pi^{0}$ proceeds dominantly via a resonance structure observed in the total cross section 
at $\sqrt{s} = 2.37\ $ GeV ~\cite{wasa1}.


Nearly all possible decay channels have been investigated ~\cite{wasa2, wasa3, wasa4}.
New data on polarised neutron-proton scattering in the region of interest exhibit a resonance pole in 
the partial waves analysis in accordance with the resonance hypothesis ~\cite{wasa5,wasa6}.
This provides the first solid evidence for the existence of a non-trivial dibaryon.

Since the measurements suggest this resonance to decay dominantly via an intermediate $\Delta \Delta$
system, it constitutes asymptotically a $\Delta \Delta$ system bound by nearly 100 MeV \ -- as 
predicted by Dyson and Xuong ~\cite{dysonxuong} already in 1964.
Furthermore, most recent relativistic three-body calculations based on hadron dynamics ~\cite{haddin}
as well as quark model calculations ~\cite{dsqm1,dsqm2} succeeded to predict properly a number
of characteristics of this resonance.

%
%
\section{Dibaryon binding mechanism} \label{sec:2.1}

In 1977 Jaffe predicted the existence of the so-called $H$ dibaryon, a hadronically bound 
$\Lambda \Lambda$ system containing two strange quarks.
After the Jaffe work, many other theoretical predictions of dibaryons appeared in the following years. 
They were based on QCD-inspired models like bag, potential, string or flux-tube models for the
six-quark system ~\cite{dsinevitable,dibpred1,dibpred2,dibpred3,dibpred5,dibpred6,dibpred7,dibpred8,
dibpred9,dibpred10,dibpred11,dibpred12,dibpred13}.
Other models were based on the hadronic baryon–baryon interaction without any explicit quark degrees
of freedom ~\cite{dibpred14,dibpred15,dibpred16} or just symmetry considerations based on SU(3)
~\cite{dibpred17}.

In general, the QCD-inspired model calculations are based on the color-magnetic interaction between 
quarks giving rise to hyperfine splitting:
\begin{equation}
    v_{color-mag} = - \sum_{i<j} (\lambda_{i} \cdot \lambda_{j}) (\sigma_{i} \cdot \sigma_{j})
    v(r_{ij})
\end{equation}
where $\lambda_{i}$ and $\sigma_{i}$ denote color and spin operators, respectively, of the quark $i$, 
while $v(r_{ij})$ is a flavor conserving short-range interaction between the quarks i and j 
~\cite{colormag}.
