% !TEX root = ../main.tex
%
\chapter{Dibaryons} \label{sec:2}

\cleanchapterquote{I think I can safely say that nobody understands quantum mechanics}{Richard P. Feynman}{(The Character of Physical Law)}


The term dibaryon denotes any object with baryon number B = 2. In this sense the first known dibaryon
has been the deuteron discovered in 1932~\cite{deu_disc}.
In terms of quarks a dibaryon is a composed state of six valence quarks.
It may be of molecular type, i.e. spatially extended with two well separated interacting quark 
bags as is the case of the deuteron. Or \ -- more exotic and hence more interesting -- \ a dibaryon
could be a spatially compact hexaquark object, where all six quarks are confined in a single quark bag.

The dibaryon searches have started in the fifties and have been very changeful, with many 
successes and failures. 
Early predictions of a vast number of dibaryon states initiated endless experimental claims,
but finally none survived careful experimental investigations. 

Despite their long history dibaryon searches have recently received renewed interest,
in particular by the recognition that there are more complex quark configurations than just the 
well known $q\bar{q}$ and $qqq$ systems \ -- e.g. the LHCb observation of a pentaquark 
state~\cite{lhcbpenta}.
The WASA-at-COSY collaboration has found that the double-pionic fusion reaction $pn \rightarrow 
d \pi^{0} \pi^{0}$ proceeds dominantly via a resonance structure observed in the total cross section 
at $\sqrt{s} = 2.37\ $ GeV~\cite{wasa1}. 


Nearly all possible decay channels have been investigated~\cite{wasa2, wasa3, wasa4}.
New data on polarised neutron-proton scattering in the region of interest exhibit a resonance pole in 
the partial waves analysis in accordance with the resonance hypothesis~\cite{wasa5,wasa6}.
This provides the first solid evidence for the existence of a non-trivial dibaryon.

Since the measurements suggest this resonance to decay dominantly via an intermediate $\Delta \Delta$
system, it constitutes asymptotically a $\Delta \Delta$ system bound by nearly 100 MeV \ -- as 
predicted by Dyson and Xuong~\cite{dysonxuong} already in 1964.
Furthermore, most recent relativistic three-body calculations based on hadron dynamics~\cite{haddin}
as well as quark model calculations~\cite{dsqm1,dsqm2} succeeded to predict properly a number
of characteristics of this resonance.

%
%
\section{Dibaryon binding mechanism} \label{sec:2.1}

In 1977 Jaffe predicted the existence of the so-called $H$ dibaryon, a hadronically bound 
$\Lambda \Lambda$ system containing two strange quarks.
After the Jaffe work, many other theoretical predictions of dibaryons appeared in the following years. 
They were based on QCD-inspired models like bag, potential, string or flux-tube models for the
six-quark system~\cite{dsinevitable1,dibpred1,dibpred2,dibpred3,dibpred5,dibpred6,dibpred7,dibpred8,
dibpred9,dibpred10,dibpred11,dibpred12,dibpred13}.
Other models were based on the hadronic baryon–baryon interaction without any explicit quark degrees
of freedom~\cite{dibpred14,dibpred15,dibpred16} or just symmetry considerations based on SU(3)
~\cite{dibpred17}.

%
% \subsection{QCD-inspired models} \label{sec:2.1.1}

In general, the QCD-inspired model calculations are based on the colour-magnetic interaction between 
quarks giving rise to hyperfine splitting:
\begin{equation} \label{eq:colmag}
    v_{colour-mag} = - \sum_{i<j} (\lambda_{i} \cdot \lambda_{j}) (\sigma_{i} \cdot \sigma_{j})
    v(r_{ij})
\end{equation}
where $\lambda_{i}$ and $\sigma_{i}$ denote colour and spin operators, respectively, of the quark $i$, 
while $v(r_{ij})$ is a flavor conserving short-range interaction between the quarks i and 
j~\cite{colormag}.
As shown by Jaffe, the colour-magnetic interaction is most attractive in the flavor-singlet state with
$I(J^{P}) = 0(0^{+})$ and $S = -2$, that represent the $H$ dibaryon. The wavefunction of this state
correspond, asymptotically, to baryon-baryon configurations of $\Lambda \Lambda$, $\Xi N$ and 
$\Sigma \Sigma$~\cite{oka}. In this picture the leading dibaryon candidates, having the underlying 
baryons in relative S wave are the following states:
\begin{itemize}
    \item $S=0 \ $, $I(J^{P}) = 0(3^{+})$ and BB structure $\Delta \Delta$,
    \item $S=-1\ $, $I(J^{P}) = 1/2(2^{+})$ and BB structure $\Sigma^{*}N$ and $\Sigma \Delta$,
    \item $S=-2\ $, $I(J^{P}) = 0(0^{+})$ and BB structure $\Lambda \Lambda$, $\Xi N$ and 
$\Sigma \Sigma$ \ -- the $H$ dibaryon,
    \item $S=-3\ $, $I(J^{P}) = 1/2(2^{+})$ and BB structure $\Omega N$, $\Xi^{*} \Sigma$, 
    $\Xi \Lambda^{*}$ and $\Xi \Sigma^{*}$,
\end{itemize}
where BB means the baryon–baryon configuration, which is asymptotically closest to the dibaryon
state.

The results based on Equation~\ref{eq:colmag} assume unbroken SU(3) symmetry.
However, more realistic models, which account for symmetry breaking, lead to significant changes
in the predicted dibaryon masses.
With respect to a possible experimental observation of dibaryon resonances, predictions of states
with low masses are particular interesting because of their expected narrow width that are easier
to observe.

Other models are based on the partial-wave analyses of $NN$ scattering.
introducing an elongated quark bag allowing for finite orbital angular momenta between
delocalised quark clusters in the partitions $q^{2}-q^{4}$ and $q-q^{5}$. 
These models derive and effective potential \ -- as the Nijmegen 
potential~\cite{dibpred1,dibpred2,dibpred3} -- \ predicting a multitude of dibaryon resonances
both in non-strange and strange sectors. 

The Los Alamos theory group predicted dibaryons in the $\Omega N$ system, which could be so
deeply bound that they would be stable with respect to strong 
decay~\cite{dsinevitable1,dsinevitable2}.
They also showed that predictions about the binding energy of the dibaryons critically depend
on the detailed dynamics of the model under consideration.
They rather emphasised the particular importance of the \textit{"inevitable"} \ds dibaryon, 
as they called it~\cite{dsinevitable1}.
They also argued that certain basic features common to all models based on one-gluon exchange
and confinement lead unavoidably to the prediction of a non-strange dibaryon resonance \ds 
with $I(J^{P}) = 0(3^{+})$ due to its special symmetry.
In the Los Alamos calculations it appears to be very deeply bound by nearly 400 MeV.
In contrast, the MIT and cloudy bag model calculations~\cite{dibpred2,dibpred5,dibpred6}
obtained for it binding energies relative to the
$\Delta \Delta$ threshold of about 100 MeV.

Recently the question, whether the H dibaryon is bound or not, has been addressed by two 
state-of-the-art lattice QCD calculations~\cite{Hlattice1,Hlattice2} obtaining
that the $H$ dibaryon to be bound by about 8 MeV.
Subsequent theoretical investigations show that such predictions require still quite some
fine-tuning~\cite{Hlattice3}.

The prediction of a copious number of dibaryon states in strange and non-strange sectors 
initiated a rush of experimental searches for such states. 
The importance of the observation of these states lies in the possibility to discriminate
between the multitude of models that try to describe the baryon-baryon binding mechanism.
Furthermore, this subject is connected to the low energy quark-quark interaction problem,
therefore the characterisation of these very interesting objects can improve our 
understanding of the non-perturbative QCD.

%
%
\section{The \dst dibaryon} \label{sec:2.2}

The \ds dibaryon \ -- or more accurately \dst dibaryon -- \ is the main subject of this thesis.
It has been observed for the first time \ as already mentioned -- \ by the WASA-at-COSY 
collaboration in 2011~\cite{wasa1}.
Event though this observation is based on solid experimental evidence, it needs an independent
confirmation.