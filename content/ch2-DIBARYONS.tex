% !TEX root = ../main.tex
%
\chapter{Dibaryons} \label{sec:2}


The dibaryon denotes any object with baryon number B = 2. In this sense the first known dibaryon
has been the deuteron discovered in 1932 ~\cite{deu_disc}.
In terms of quarks a dibaryon is composed of six valence quarks.
It may be of molecular type, i.e. spatially extended with two well separated interacting quark 
bags as is the case for the deuteron. Or \ -- more exotic and hence more interesting -- \ a dibaryon
could be a spatially compact hexaquark object, where all six quarks are confinde in a single quark bag.

The dibaryon searches have started in the fifties and have been a very changeful one with many 
successes and failures. 
Early predictions of a vast number of dibaryon states initiated endless experimental claims,
but finally none survived careful experimental investigations. 

Despite their long history dibaryon searches have recently received renewed interest,
in particular by the recognition that there are more complex quark configurations than just the well known
$q\bar{q}$ and $qqq$ systems.
The WASA-at-COSY collaboration has found that the double-pionic fusion reaction $pn \rightarrow 
d \pi^{0} \pi^{0}$ proceeds dominantly via a resonance structure observed in the total cross section at 
$sqrt{s} = 2.37\ $ GeV ~\cite{wasa1}.


Nearly all possible decay channels have been investigated ~\cite{wasa2, wasa3, wasa4}.
New data on polarised neutron-proton scattering in the region of interest exhibit a resonance pole in 
the partial waves analysis in accordance with the resonance hypothesis ~\cite{wasa5,wasa6}.
This provides the first solid evidence for the existence of a non-trivial dibaryon.

Since the measurements suggest this resonance to decay dominantly via an intermediate $\Delta \Delta$ system,
it constitutes asymptotically a $\Delta \Delta$ system bound by nearly 100 MeV \ -- as predicted by Dyson and
Xuong ~\cite{dysonxuong} already in 1964.


% as predicted by Dyson and Xuong [33] already in 1964 and later- on also by Goldman et al. [34], who called it the ‘‘inevitable dibaryon’’ d∗ due to its unique symmetry features. Most recent relativistic three-body calculations based on hadron dynamics [35,36] as well as quark model calculations [37–39] succeeded to predict properly a number of characteristics of this resonance. The latter also postulate a substantial hidden- color component accounting in particular for the unusually narrow width of this resonance.